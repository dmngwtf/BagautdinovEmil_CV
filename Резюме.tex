\documentclass[a4paper,11pt]{article}
\usepackage[margin=0.7in]{geometry}
\usepackage[utf8]{inputenc}
\usepackage[T2A]{fontenc}
\usepackage[russian]{babel}
\usepackage{titlesec}
\usepackage{enumitem}
\usepackage{xcolor}
\usepackage{hyperref}


\hypersetup{
	colorlinks=true,
	urlcolor=blue
}

% Компактные секции и списки
\titlespacing*{\section}{0pt}{6pt}{4pt}
\setlist[itemize]{leftmargin=1.5em, itemsep=2pt, topsep=2pt, parsep=0pt, partopsep=0pt}
\renewcommand{\labelitemi}{--}

\titleformat{\section}{\large\bfseries}{}{0pt}{}
\titleformat{\subsection}{\normalsize\bfseries}{}{0pt}{}

\begin{document}
	
	\begin{center}
		{\Large\textbf{Багаутдинов Эмиль Айратович}}\\[3pt]
		\normalsize
		Тел.: +7 (903) 341-49-80 \quad
		Email: \href{mailto:emlg1703@gmail.com}{emlg1703@gmail.com} \\
		Telegram: \href{https://t.me/e_eemil}{\texttt{@e\_eemil}}
		GitHub: \href{https://github.com/dmngwtf}{github.com/dmngwtf}
	\end{center}
	
	\vspace{4pt}
	
	\section*{Профиль}
	Junior Backend Developer / Python.  
	Опыт разработки REST API на FastAPI и PostgreSQL, написания асинхронного кода на asyncio, контейнеризации сервисов в Docker и нагрузочного тестирования (Locust).  
	Цель — развиваться в разработке бэкенда и создании надёжных производительных сервисов.
	
	\section*{Образование}
	\textbf{НИУ ВШЭ}, Москва — ФКН, Программная инженерия (до 2027 г.)
	\section*{Опыт работы}
	
	\subsection*{Tele2 — Backend Developer, команда студенческих проектов (декабрь 2024 — настоящее время)}
	
	\textit{Python, FastAPI, ClickHouse, KeyDB, SQL, Docker, REST API, Grafana, Dash, Git, asyncio}
	
Разрабатываю бэкенд для внутреннего сервиса Tele2, показывающего базовые станции и покрытие России. Реализую REST API для геоданных, интеграцию с ClickHouse и KeyDB, обеспечиваю визуализацию через Grafana и Dash. Оптимизирую SQL, кэширую данные и поддерживаю стабильность при высокой нагрузке. Участвую в архитектуре и деплое контейнеризированных Python-сервисов.

	
	
	
	\section*{Ключевые навыки}
	
	\textbf{Языки:} Python, C++, C\\
	\textbf{Бэкенд:} FastAPI, SQLAlchemy, Pydantic, asyncio\\
	\textbf{БД:} PostgreSQL, TimeScaleDB, Redis, MongoDB\\
	\textbf{Инфраструктура:} Docker, Kubernetes, Pytest, Locust, Kafka, Git\\
	\textbf{Языки:} русский — родной; английский — C1
	
	
	\section*{Пет-проекты}
	
	\subsection*{Бэкенд-сервис пункта выдачи заказов}
	\textit{FastAPI, PostgreSQL, SQLAlchemy, Docker, Pytest}  
	Разработал REST API для учёта и выдачи заказов. Настроил миграции и контейнеризацию сервиса, покрыл тестами > 80 \%. \\ 
	\href{https://github.com/dmngwtf/backend_pvz_service}{github.com/dmngwtf/backend\_pvz\_service}
	
	\subsection*{Auth \& Orders Service}
	\textit{FastAPI, AsyncPG, RabbitMQ, PostgreSQL, Docker, Alembic}
	
	Разработал асинхронный сервис для управления пользователями и заказами: JWT-аутентификация, обработка заказов через очереди, автоматические миграции и контейнеризация. \\
	\href{https://github.com/dmngwtf/auth-order-serviceo}{https://github.com/dmngwtf/auth-order-service}
	

	
\subsection*{BotFarm Service}
\textit{FastAPI, AsyncPG, SQLAlchemy, PostgreSQL, Docker, Kubernetes, Alembic}

Разработал асинхронный микросервис для управления фермой ботов с REST API и JWT-аутентификацией администраторов. Реализовал CRUD для аккаунтов ботов, управление проектами и окружениями, автоматические миграции базы данных и контейнеризацию с поддержкой Kubernetes. \\
\href{https://github.com/dmngwtf/botfarm_service}{https://github.com/dmngwtf/botfarm\_service}



	
	
	
	
	
	
	
\end{document}