\documentclass[a4paper,11pt]{article}
\usepackage[margin=0.7in]{geometry}
\usepackage[utf8]{inputenc}
\usepackage[T2A]{fontenc}
\usepackage[russian]{babel}
\usepackage{titlesec}
\usepackage{enumitem}
\usepackage{xcolor}
\usepackage{hyperref}

\hypersetup{
	colorlinks=true,
	urlcolor=blue
}

% Компактные секции и списки
\titlespacing*{\section}{0pt}{6pt}{4pt}
\setlist[itemize]{leftmargin=1.5em, itemsep=2pt, topsep=2pt, parsep=0pt, partopsep=0pt}
\renewcommand{\labelitemi}{--}

\titleformat{\section}{\large\bfseries}{}{0pt}{}
\titleformat{\subsection}{\normalsize\bfseries}{}{0pt}{}

\begin{document}
	
	\begin{center}
		{\Large\textbf{Багаутдинов Эмиль Айратович}}\\[3pt]
		\normalsize
		Тел.: +7 (903) 341-49-80 \quad
		Email: \href{mailto:emil.bashautdinov@yandex.ru}{emil.bashautdinov@yandex.ru} \\
		Telegram: \href{https://t.me/e_eemil}{@e_eemil} \quad
		GitHub: \href{https://github.com/dmngwtf}{github.com/dmngwtf}
	\end{center}
	
	\vspace{4pt}
	
	\section*{Профиль}
	Junior Backend Developer / Python.  
	Опыт разработки REST API на FastAPI и PostgreSQL, написания асинхронного кода на asyncio, контейнеризации сервисов в Docker и нагрузочного тестирования (Locust).  
	Цель — развиваться в разработке бэкенда и создании надёжных производительных сервисов.
	
	\section*{Образование}
	\textbf{НИУ ВШЭ}, Москва — ФКН, Программная инженерия (до 2027 г.)
	\section*{Опыт работы}
	
	\subsection*{Мастерская телекома — Backend Developer (декабрь 2024 — настоящее время)}
	\textit{FastAPI, PostgreSQL, SQLAlchemy, Docker, asyncio, GIS APIs}
	
	Разрабатываю бэкенд-инфраструктуру для внутреннего сервиса крупного телекома, отображающего карту покрытия базовых станций.  
	Реализую REST API для обработки и визуализации геоданных, интеграцию с внешними GIS-сервисами и базой данных вышек.  
	Оптимизирую SQL-запросы, добавляю кэширование и обеспечиваю стабильную работу сервиса при высокой нагрузке.  
	Участвую в проектировании архитектуры и деплое контейнеризированных сервисов в Docker.
	
	\section*{Ключевые навыки}
	\textbf{Языки:} Python (async, generators, OOP), C++, C  
	\textbf{Бэкенд:} FastAPI, SQLAlchemy, Pydantic, asyncio  
	\textbf{БД:} PostgreSQL, TimeScaleDB, Redis, MongoDB  
	\textbf{Инфраструктура:} Docker, Pytest, Locust, Kafka, Git  
	\textbf{NLP:} TF-IDF, Word2Vec, Sentence-BERT, Transformers (HuggingFace)  
	\textbf{Алгоритмы:} структуры данных, оптимизация поиска, мат. анализ, дискретка  
	\textbf{Языки:} русский — родной; английский — B2+
	
	\section*{Проекты}
	
	\subsection*{Бэкенд-сервис пункта выдачи заказов}
	\textit{FastAPI, PostgreSQL, SQLAlchemy, Docker, Pytest}  
	Разработал REST API для учёта и выдачи заказов. Настроил миграции и контейнеризацию сервиса, покрыл тестами > 80 \%.  
	\href{https://github.com/dmngwtf/backend_pvz_service}{github.com/dmngwtf/backend\_pvz\_service}
	
	\subsection*{Микросервис для сокращения ссылок}
	\textit{FastAPI, Redis, PostgreSQL, Alembic, Docker}  
	Реализовал хеширование URL, кэширование в Redis и миграции БД. Подготовил контейнер для развёртывания.  
	\href{https://github.com/dmngwtf/url_generator_api}{github.com/dmngwtf/url\_generator\_api}
	
	\subsection*{TCP-сервер для сбора метрик}
	\textit{Python, asyncio, TimescaleDB, structlog, Pytest}  
	Реализовал приём метрик по TCP и сохранение во временные ряды в TimescaleDB. Настроил логирование и модульные тесты.  
	\href{https://github.com/dmngwtf/metric_collector}{github.com/dmngwtf/metric\_collector}
	
	\subsection*{TCP-ECHO бенчмарк}
	\textit{Python, asyncio, threading, matplotlib}  
	Исследовал производительность асинхронной и многопоточной моделей, сравнил RPS и Latency.  
	\href{https://github.com/dmngwtf/highload_tcp_echo_server_benchmark}{github.com/dmngwtf/highload\_tcp\_echo\_benchmark}
	
	\subsection*{ИИ-ассистент в Telegram-боте}
	\textit{Python, Aiogram, Vosk, gTTS, FFmpeg, asyncio}  
	Разработал бота для распознавания и синтеза речи, интеграция с API медицинских сервисов.  
	\href{https://github.com/dmngwtf/AI_assistant_doctor}{github.com/dmngwtf/AI\_assistant\_doctor}
	
	\subsection*{Бэкенд-сервис магазина с мерчем}
	\textit{FastAPI, PostgreSQL, Docker, Pytest, Locust}  
	Реализовал авторизацию и покупки, провёл нагрузочные тесты (до 500 RPS).  
	\href{https://github.com/dmngwtf/avito_merchshop}{github.com/dmngwtf/avito\_merchshop}
	
	\section*{Дополнительное обучение}
	\begin{itemize}
		\item \textit{Fluent Python} — продвинутые концепции Python
		\item Э. Танненбаум «Компьютерные сети» — принципы и протоколы
	\end{itemize}
	
\end{document}
