\documentclass[a4paper,11pt]{article}
\usepackage[margin=0.7in]{geometry}
\usepackage[utf8]{inputenc}
\usepackage[T2A]{fontenc}
\usepackage[russian]{babel}
\usepackage{titlesec}
\usepackage{enumitem}
\usepackage{xcolor}
\usepackage{hyperref}

\hypersetup{
	colorlinks=true,
	urlcolor=blue
}

% Компактные секции и списки
\titlespacing*{\section}{0pt}{6pt}{4pt}
\setlist[itemize]{leftmargin=1.5em, itemsep=2pt, topsep=2pt, parsep=0pt, partopsep=0pt}
\renewcommand{\labelitemi}{--}

\titleformat{\section}{\large\bfseries}{}{0pt}{}
\titleformat{\subsection}{\normalsize\bfseries}{}{0pt}{}

\begin{document}
	
	\begin{center}
		{\Large\textbf{Багаутдинов Эмиль Айратович}}\\[3pt]
		\normalsize
		Тел.: +7 (903) 341-49-80 \quad
		Email: \href{mailto:emlg1703@gmail.com}{emlg1703@gmail.com} \\
		Telegram: \href{https://t.me/e_eemil}{\texttt{@e\_eemil}}
		GitHub: \href{https://github.com/dmngwtf}{github.com/dmngwtf}
	\end{center}
	
	\vspace{4pt}
	
	\section*{Профиль}
	Junior Backend Developer / Python.  
	Опыт разработки REST API на FastAPI и PostgreSQL, написания асинхронного кода на asyncio, контейнеризации сервисов в Docker и нагрузочного тестирования (Locust).  
	Цель — развиваться в разработке бэкенда и создании надёжных производительных сервисов.
	
	\section*{Образование}
	\textbf{НИУ ВШЭ}, Москва — ФКН, Программная инженерия (до 2027 г.)
	\section*{Опыт работы}
	
	\subsection*{Tele2 — Backend Developer, команда студенческих проектов (декабрь 2024 — настоящее время)}
	
	\textit{Python, FastAPI, ClickHouse, KeyDB, SQL, Docker, REST API, Grafana, Dash, Git, asyncio}
	
	Разрабатываю бэкенд-инфраструктуру внутреннего сервиса Tele2 для инженеров, который отображает базовые станции и покрытие по всей России на основе ежедневных отчетов из базы данных.  
	Реализую REST API для обработки и подготовки геоданных, интеграцию с ClickHouse и KeyDB, а также обеспечение данных для визуализации через Grafana и фронтенд на Dash.  
	Оптимизирую SQL-запросы, добавляю кэширование и гарантирую стабильную работу бэкенда при высокой нагрузке.  
	Участвую в проектировании архитектуры и деплое контейнеризированных Python-сервисов в Docker.
	
	
	
	\section*{Ключевые навыки}
	
	\textbf{Языки:} Python, C++, C\\
	\textbf{Бэкенд:} FastAPI, SQLAlchemy, Pydantic, asyncio\\
	\textbf{БД:} PostgreSQL, TimeScaleDB, Redis, MongoDB\\
	\textbf{Инфраструктура:} Docker, Kubernetes, Pytest, Locust, Kafka, Git\\
	\textbf{Языки:} русский — родной; английский — C1
	
	
	\section*{Пет-проекты}
	
	\subsection*{Бэкенд-сервис пункта выдачи заказов}
	\textit{FastAPI, PostgreSQL, SQLAlchemy, Docker, Pytest}  
	Разработал REST API для учёта и выдачи заказов. Настроил миграции и контейнеризацию сервиса, покрыл тестами > 80 \%. \\ 
	\href{https://github.com/dmngwtf/backend_pvz_service}{github.com/dmngwtf/backend\_pvz\_service}
	
	\subsection*{Универсальный Telegram-бот для скачивания видео}
	\textit{Python, aiogram, PostgreSQL/TimescaleDB, Docker, diskcache, Pytest}
	Реализовал асинхронный Telegram-бот для скачивания видео, инегрировал распознавание музыки, кеширование, доступ к истории скачиваний для пользователя. \\
	\href{https://github.com/dmngwtf/tiktok-siphon}{https://github.com/dmngwtf/tiktok-siphon}
	
	\subsection*{TCP-сервер для сбора метрик}
	\textit{Python, asyncio, TimescaleDB, structlog, Pytest}  
	Реализовал приём метрик по TCP и сохранение во временные ряды в TimescaleDB. Настроил логирование и модульные тесты.  \\
	\href{https://github.com/dmngwtf/metric_collector}{github.com/dmngwtf/metric\_collector}
	
	\subsection*{TCP-ECHO бенчмарк}
	\textit{Python, asyncio, threading, matplotlib}  
	Исследовал производительность асинхронной и многопоточной моделей, сравнил RPS и Latency.  
	\href{https://github.com/dmngwtf/highload_tcp_echo_server_benchmark}{github.com/dmngwtf/highload\_tcp\_echo\_benchmark}
	
	
	
\end{document}