\documentclass[a4paper,12pt]{article}
\usepackage[margin=1in]{geometry}
\usepackage{titlesec}
\usepackage{enumitem}
\usepackage{hyperref}
\usepackage[russian]{babel}
\usepackage[T2A]{fontenc}
\usepackage[utf8]{inputenc}
\usepackage{xcolor}
\usepackage{hyperref}
\hypersetup{
	colorlinks=true,
	urlcolor=blue
}


\titleformat{\section}{\large\bfseries}{}{0pt}{}
\titleformat{\subsection}{\normalsize\bfseries}{}{0pt}{}
\renewcommand{\labelitemi}{--}

\begin{document}
	
	\begin{center}
		\Large\textbf{Багаутдинов Эмиль Айратович} \\
		\normalsize
		\vspace{5pt}
		\begin{tabular}{rl}
			Email: & \href{mailto:emil.bashautdinov@yandex.ru}{emil.bashautdinov@yandex.ru} \\
			GitHub: & \href{https://github.com/dmngwtf}{github.com} \\
		\end{tabular}
	\end{center}
	
	\vspace{10pt}
	
	\section*{Образование}
	\textbf{НИУ ВШЭ}, Москва \\
	Факультет: МИЭМ — Московский институт электроники и математики \\
	Направление: Инфокоммуникационные технологии и системы связи
	
	\vspace{10pt}
	
	\section*{Навыки}
	
	\subsection*{Языки программирования}
	\begin{itemize}[leftmargin=1.5em]
		\item C, C++, Python
	\end{itemize}
	
	\subsection*{Web и Backend}
	\begin{itemize}[leftmargin=1.5em]
		\item FastAPI (создание REST API)
		\item SQLAlchemy, Pydantic (ORM и валидация)
		\item Aiogram (Telegram-боты)
		\item asyncio (асинхронное программирование)
	\end{itemize}
	
	\subsection*{Базы данных}
	\begin{itemize}[leftmargin=1.5em]
		\item PostgreSQL
		\item TimeScaleDB
		\item MongoDB
		\item Проектирование структуры БД, оптимизация запросов
	\end{itemize}
	
	\subsection*{Сети}
	\begin{itemize}
		\item Широкая теоретическая база на основе трудов Э. Танненбаума
		\item Модель OSI и стек TCP/IP
		\item ARP, NAT, маршрутизация, IP (v4/v6), DNS, DHCP
		\item Протоколы: TCP, UDP, HTTP, FTP, SMTP, SNMP
		\item Основы сетевой безопасности, работа коммутаторов и маршрутизаторов
	\end{itemize}
	
	\subsection*{Аудио и голосовые технологии}
	\begin{itemize}[leftmargin=1.5em]
		\item Vosk, SpeechRecognition (распознавание речи)
		\item FFmpeg (обработка аудио)
	\end{itemize}
	
	\subsection*{Инфраструктура и DevOps}
	\begin{itemize}[leftmargin=1.5em]
		\item Docker (контейнеризация, настройка окружения)
		\item Locust (нагрузочное тестирование)
		\item Pytest (юнит-тестирование)
		\item Redis (Кеширование)
	\end{itemize}
	
	\subsection*{Алгоритмы и математика}
	\begin{itemize}[leftmargin=1.5em]
		\item Знание алгоритмов и структур данных
		\item Призовое место в тренировках по алгоритмам от яндекса 5.0
		\item Участие в олимпиадном программировании
		\item Два курса высшей математики НИУ ВШЭ
		
	\end{itemize}
	
	\subsection*{Языки}
	\begin{itemize}[leftmargin=1.5em]
		\item Русский — родной
		\item Английский — B2+
	\end{itemize}
	
	\vspace{10pt}
	
	
	
	
	\section*{Проекты}
	
	\subsection*{Бэкенд-сервис пункта выдачи заказов}
	Сервис управления пунктами выдачи и складом \\
	\textit{FastAPI, PostgreSQL, SQLAlchemy, Docker, Pydantic, Pytest}\\  
	REST API, учёт товаров, контейнеризация, тесты\\
	{Код: \href{https://github.com/dmngwtf/backend_pvz_service}{\texttt{github.com/dmngwtf/backend\_pvz\_service}}}

		
		
	
	\subsection*{Микросервис для сокращения ссылок}
	Сокращение и перенаправление URL  \\
	\textit{Python, FastAPI, SQLAlchemy, Alembic, Redis, PostgreSQL, Docker} \\ 
	Хеширование, хранение в БД, кэширование, миграции\\
	{Код: \href{https://github.com/dmngwtf/url_generator_api}{\texttt{github.com/dmngwtf/url\_generator\_api}}
	
	
	\subsection*{TCP-сервер для сбора метрик}
	Сбор, хранение и запись метрик  TimescaleD \\
	\textit{Python, asyncio, structlog , PostgreSQL, TimescaleDB, Pytest} \\ 
	Сбор, хранение, запись метрик в БД, миграции\\
	{Код: \href{https://github.com/dmngwtf/metric_collector}{\texttt{github.com/dmngwtf/metric\_collector}}
		
	\subsection*{TCP-ECHO бенчмарк}
	Сравнение асинхронной и многопоточной моделей сервера\\
	\textit{Python, asyncio, threading, matpotlib, tqdm} \\ 
	Сбор и визуализация метрик производительности (RPS, Latency, Memory Usage)\\
	{Код: \href{https://github.com/dmngwtf/highload_tcp_echo_server_benchmark}{\texttt{github.com/dmngwtf/highload\_tcp\_echo\_benchmark}}
		
	
	
	\subsection*{ИИ-ассистент в Telegram-боте}
	Бот для голосовой диагностики  \\
	\textit{Python, aiogram, Vosk, gTTS, asyncio, FFmpeg} \\ 
	Распознавание речи, синтез ответов, API медицины\\
	{Код: \href{https://github.com/dmngwtf/AI_assistant_doctor}{\texttt{github.com/dmngwtf/AI\_assistant\_doctor}}
	
	
	\subsection*{Прошивка для полётного контроллера (в команде)}
	ПО для российского микроконтроллера MIK32  \\
	\textit{C, RTOS, Betaflight}  \\
	Сенсоры, логика управления, низкоуровневая разработка\\
	Код: \textit{располагается в закрытом репозитории университета}
	
	\subsection*{Бэкенд-сервис магазина с мерчем}
	Интернет-магазин (тестовое задание)  \\
	\textit{FastAPI, PostgreSQL, Docker, Locust, Pytest}  \\
	Авторизация, покупки, нагрузочное тестирование\\
	{Код: \href{https://github.com/dmngwtf/avito_merchshop}{\texttt{github.com/dmngwtf/avito\_merchshop}}
	
	
	\subsection*{Матричный калькулятор}
	GUI-калькулятор матриц  \\
	\textit{C++, Qt}  \\
	Собственная реализация операций, университетский проект\\
	Код: \textit{располагается в закрытом репозитории университета}
	
	
	
	\section*{Цель}
	Хочу начать карьеру в IT-компании, где смогу применить свои знания в backend-разработке, проектировании архитектуры, интеграции сервисов и работе с данными. Интересуюсь высоконагруженными системами, микросервисами и голосовыми интерфейсами. Готов быстро учиться и приносить практическую пользу.
	
\end{document}
